Sustainability focuses on the endurance of processes and products.  It is perhaps most widely associated with environmental science and climate science. All engineering disciplines are involved in sustainability initiatives, e.g., design for the environment. Sustainability is an inherent challenge in modern systems and software engineering. 

\textit{Sustainability engineering} is the discipline of constructing systems that support and enable sustainability. In sustainability engineering, many different kinds of models have to be integrated. Engineering models (which are used in development) need to be combined with scientific models (which  promote understanding of sustainability concerns, and help underpin decision making). Traditionally, integration of such models has been carried out by engineers who use scientific models as constraints.  This approach is (ironically) unsustainable, and as our ambition and the complexity of sustainable systems increases, we are encountering massive challenges in integrating different models. These challenges should be addressed by the computing community, with expertise in modelling, in close collaboration with domain experts.

This paper gives an overview of key research and an analysis of research challenges at the intersection of modeling and sustainability, which resulted from a week-long Dagstuhl seminar involving leading experts from different domains. It presents an example of how one of these research challenges might be addressed, integrating ideas from modeling, sustainability engineering and value-based engineering. The research challenges in particular will serve to establish a research agenda for collaboration between software engineers and sustainability researchers.

\section{Conclusions and Future Work}
\label{future}
In this paper we built atop our previous work presented in~\cite{zolotas2015inference} to support type inference in flexible MDE approaches, providing support for moving from partially typed example models to more complete ones. In our previous work we used a single classification algorithm, CART. In this work, we used a second algorithm based on Random Forests to assess how this will affect the prediction accuracy. The results showed that CART has already maximized the prediction performance and the use of an algorithm that belongs to the same category does not improve the results. For the Random Forests algorithm we used 7 different values for the number of trees that the algorithm is trained with, identifying a point (50 trees) after which the prediction accuracy reaches a plateau. In addition, in this work we injected noise in 4 out of the 5 variables used by creating more ``sparse'' example models. The results showed that this has an impact in some metamodels. Finally, we calculated the importance of each variable in both algorithms.

The approach is intended to be used to support flexible modelling, where examples can be created in ways that are not restricted by metamodels. However, it could be applied directly to traditional MDE, for instance, to infer types for an already-typed model, which may potentially reveal poor or incorrect type assignments or misuses of the metamodel.

In the future, we plan to make use of additional features. Work in this direction was presented in~\cite{zolotas2015typeGraphical} where 4 spatial/graphical related features are used (i.e., colour of the node, width and height, shape). A user study in which domain experts will create real example models using a flexible MDE approach (e.g., Muddles) is of interest. This will also allow the combination of the four features based on spatial characteristics mentioned above with those presented in this work. This is not possible at this point as \textbf{all} the nodes of the synthetic muddles created as part of this work have exactly the same graphical characteristics (i.e., shape, colour and dimensions).

In addition, the names that the domain experts choose to assign to the semantic characteristics could also be assessed to improve the predictions (e.g., the name of the attributes of each class). Initial work in this direction has been carried out in~\cite[Chapter~5]{zolotasThesis} using a widely used similarity algorithm, called Similarity Flooding~\cite{melnik2002similarity}. Results suggest that a combination with the approaches presented in this work is possible and could improve prediction results. As mentioned before, we base this work on the assumption that domain experts may use different naming conventions to express the same structural information. However, we could overcome this by assigning weights to the importance of name-matching feature: if the examples are generated by more than one domain experts then decrease the impact of the name matching in the prediction. 


